\PassOptionsToPackage{svgnames,dvipsnames}{xcolor}

\documentclass[12pt]{cmuthesis}

\usepackage[Lenny]{fncychap}
\ChNameVar{\Large}

\input{sections/packages}
\input{sections/macros}

% \includeonly{sections/cvxpyth}
\draftstamp{\today}{DRAFT}

\begin {document}
\frontmatter

\pagestyle{empty}

\title{{\bf Differentiable Optimization-Based Modeling for Machine Learning}}
\author{Brandon Amos}
\date{May 2019}
\Year{2019}
\trnumber{CMU-CS-19-X}

\committee{
\begin{tabular}{rl}
J. Zico Kolter, Chair & \textit{Carnegie Mellon University} \\
Barnab{\'a}s P{\'o}czos & \textit{Carnegie Mellon University} \\
Jeff Schneider & \textit{Carnegie Mellon University} \\
Nando de Freitas & \textit{DeepMind} \\
Vladlen Koltun & \textit{Intel Labs} \\
\end{tabular}
}

\support{
  % This thesis was supported by the National Science Foundation Graduate
  % Research Fellowship Program under Grant No.~DGE1252522.
}
\disclaimer{}

\keywords{machine learning, statistical modeling,
  convex optimization, deep learning, control,
  reinforcement learning}

\maketitle

\begin{dedication}
  To all of the people that light up my life. {\ensuremath\heartsuit}
\end{dedication}

\begin{abstract}
  Domain-specific modeling priors and specialized components are
  becoming increasingly important to the machine learning field.
  These components integrate specialized knowledge that we have
  as humans into model.
  We argue in this thesis that optimization methods provide an
  expressive set of operations that should be part of the
  machine learning practitioner's modeling toolbox.

  We present two foundational approaches for optimization-based modeling:
  1) the \emph{OptNet} architecture that integrates
  optimization problems as individual layers in larger end-to-end
  trainable deep networks, and
  2) the \emph{input-convex neural network (ICNN)}
  architecture that helps make inference and learning in deep
  energy-based models and structured prediction more tractable.

  We then show how to use the OptNet approach
  1) as a way of combining model-free and model-based reinforcement
  learning with differentiable optimal control and
  2) for top-$k$ learning problems.
  We conclude by showing how to turn the \cvxpy domain
  specific language into a differentiable optimization layer
  that enables rapid prototyping of the approaches
  in this thesis.
\end{abstract}

\begin{acknowledgments}
  Conditional on this thesis being accepted I would like to
  make the following incomplete and unfinished
  set of acknowledgments. \vspace{6mm}

  I have been incredibly fortunate and privileged throughout
  my entire life to have been given many opportunities
  that have led me to pursue this thesis research.
  Thanks to the thousands of people in the universe throughout
  the past few millennia who have provided me with the
  foundation, environment, safety, health, support, service,
  financial well-being, love, joy, knowledge, kindness, calmness,
  and happiness to produce this work.

  This thesis would not have been possible without the close
  collaboration I have had with my advisor J.~Zico Kolter over
  the past few years.
  Zico's creativity and passion have profoundly shaped
  the way I think about academic problems and pursue
  research directions, and more broadly I have learned much
  more from him along the way.
  I am incredibly grateful for the immense
  amount of time and energy Zico has put into shaping the
  direction of this work.

  Thanks to all of my close collaborators who have contributed
  to projects appearing in this thesis, including
  Byron Boots, Ivan Jimenez, Jacob Sacks, and Lei Xu,
  and more recently
  Akshay Agrawal,
  Shane Barratt,
  Stephen Boyd,
  Steven Diamond,
  and Brendan O'Donoghue.
  Your insights and contributions have deeply
  shaped this work.

  This thesis was also made possible by the great research
  environment that CMU has provided me during my Ph.D.~studies here.
  CMU's collaborative, thriving, and understanding environment gave
  me the true capabilities to pursue my passions throughout my time here.
  I spent my first two years honing my systems skills working on
  wearable cognitive assistance applications with
  Mahadev (Satya) Satyanarayanan.
  I am indebted to Satya for
  kindly giving me the freedom to pursue my interests in
  machine learning while part of his systems group and
  hope that someday I will be able to pay this forward.
  The list of all of the other students here I have learned from
  is too long to exhaustively list out here, but includes
  Alnur Ali,
  Filipe de Avila Belbute-Peres,
  Shaojie Bai,
  Noam Brown,
  Volkan Cirik,
  Zhuo Chen,
  Jeremy Cohen,
  Travis Dick,
  Jonathan Dinu,
  Priya Donti,
  Benjamin Gilbert,
  Kiryong Ha,
  Jan Harkes,
  Wenlu Hu,
  Roger Iyengar,
  Jay-Yoon Lee,
  Chun Kai Ling,
  Gaurav Manek,
  Vaishnavh Nagarajan,
  Vittorio Perera,
  Padmanabhan (Babu) Pillai,
  George Philipp,
  Leslie Rice,
  Wolf Richter,
  Mel Roderick,
  Petar Stojanov,
  Dougal Sutherland,
  Po-Wei Wang,
  Josh Williams,
  Ezra Winston,
  Junjue Wang,
  Eric Wong,
  Han Zhao, and
  Xiao Zhang.

  My Ph.D.~would have been severely lacking without my internships
  at DeepMind in 2017 and Intel Labs in 2018.
  I learned how to craft large-scale reinforcement learning systems
  from Nando de Freitas and Misha Denil at DeepMind and
  how to architect and approach vision problems from
  Vladlen Koltun at Intel Labs.
  Thank you all for hosting me.
  I value the time that I spent as part of your groups.
  I am grateful for all of the conversations and collaborations
  with the other interns and researchers in the industry as well,
  including
  Yannis Assael,
  Serkan Cabi,
  Kris Cao,
  Qifeng Chen,
  Yutian Chen,
  Mike Chrzanowski,
  Sergio Gomez Colmenarejo,
  Tim Cooijmans,
  Soham De,
  Laurent Dinh,
  Vincent Dumoulin,
  Tom Erez,
  Michael Figurnov,
  Jakob Foerster,
  Yaroslav Ganin,
  Yang Gao,
  Caglar Gulcehre,
  Karol Hausman,
  Matthew W.~Hoffman,
  Drew Jaegle,
  David Lindell,
  Simon Kohl,
  Alistair Muldal,
  Alexander Novikov,
  Tom Le Paine,
  Ben Poole,
  Rene Ranftl,
  Scott Reed,
  German Ros,
  Evan Shelhamer,
  Casper Kaae Sønderby,
  Brendan Shillingford,
  Yuval Tassa,
  Jonathan Uesato,
  Ziyu Wang,
  Abhay Yadav,
  Xuaner Zhang, and
  Yuke Zhu.

  I am grateful to the broader machine learning research community that
  has been thriving throughout my studies, including the
  Caffe, PyTorch, and TensorFlow communities I have interacted
  with over the years.
  The conferences I've attended have been invigorating and
  people are truly collaborative and ideas are
  extremely composable.
  Thanks to Soumith Chintala, Adam Paszke, and the rest of the
  Torch/PyTorch community for helping me debug and contribute
  back to the community.
  Thanks also to
  David Belanger,
  Alex Terenin,
  Rowan Zellers,
  for graciously providing me with insights or
  debugging help.

  Looking back, my teachers and mentors earlier in my life
  ignited my interests in mathematics and computer science
  and opened my eyes.
  My high school teachers
  Suzanne Nicewonder,
  Susheela Shanta, and
  Janet Washington gave me a solid foundation
  in engineering and mathematics.
  Mack McGhee at Sunapsys hosted me for an
  internship that introduced to the wonderful
  world of Linux.
  The collaborations during my undergrad were eye-opening.
  Layne T.~Watson and David Easterling
  introduced me to the beautiful fields
  of optimization, numerical methods, and
  high-performance computing, and taught me how to
  write extremely optimized and robust Fortran code.
  I apologize for going to the dark side and writing
  ANTODL (another thesis on deep learning).
  Jules White and Hamilton Turner taught me how
  to hack Android internals and architect awesome Scala code.
  Binoy Ravindran, Alastair Murray, and Rob Lyerly
  taught me how to hack on compilers
  and the Linux kernel.

  Lastly, I would especially like to thank all of my
  other friends, family members, and partners that
  have provided me with an immense amount of love,
  support, and encouragement throughout the years,
  as well as the arts, music, yoga, meditation, cycling, dance,
  lifting, board games, theatre, poetry,
  and climbing communities in
  Pittsburgh, San Francisco, and London that have
  provided a near-infinite amount of distractions
  from this thesis work.
\end{acknowledgments}

\pagestyle{plain}

\tableofcontents
\listoffigures
\listoftables
\listofalgorithms

\mainmatter

\include{sections/intro}
\include{sections/background}

\part{Foundations}
\include{sections/optnet}
\include{sections/icnn}

\part{Extensions and Applications}
\include{sections/empc}
\include{sections/lml}
\include{sections/cvxpyth}

\part{Conclusions and Future Directions}
\include{sections/conclusions}

\printbibliography[heading=bibintoc]

\end{document}

%%% Local Variables:
%%% coding: utf-8
%%% mode: latex
%%% TeX-engine: xetex
%%% End:
